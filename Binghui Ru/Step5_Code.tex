% Options for packages loaded elsewhere
\PassOptionsToPackage{unicode}{hyperref}
\PassOptionsToPackage{hyphens}{url}
%
\documentclass[
  12pt,
]{article}
\usepackage{amsmath,amssymb}
\usepackage{setspace}
\usepackage{iftex}
\ifPDFTeX
  \usepackage[T1]{fontenc}
  \usepackage[utf8]{inputenc}
  \usepackage{textcomp} % provide euro and other symbols
\else % if luatex or xetex
  \usepackage{unicode-math} % this also loads fontspec
  \defaultfontfeatures{Scale=MatchLowercase}
  \defaultfontfeatures[\rmfamily]{Ligatures=TeX,Scale=1}
\fi
\usepackage{lmodern}
\ifPDFTeX\else
  % xetex/luatex font selection
\fi
% Use upquote if available, for straight quotes in verbatim environments
\IfFileExists{upquote.sty}{\usepackage{upquote}}{}
\IfFileExists{microtype.sty}{% use microtype if available
  \usepackage[]{microtype}
  \UseMicrotypeSet[protrusion]{basicmath} % disable protrusion for tt fonts
}{}
\makeatletter
\@ifundefined{KOMAClassName}{% if non-KOMA class
  \IfFileExists{parskip.sty}{%
    \usepackage{parskip}
  }{% else
    \setlength{\parindent}{0pt}
    \setlength{\parskip}{6pt plus 2pt minus 1pt}}
}{% if KOMA class
  \KOMAoptions{parskip=half}}
\makeatother
\usepackage{xcolor}
\usepackage[margin=2.5cm]{geometry}
\usepackage{graphicx}
\makeatletter
\newsavebox\pandoc@box
\newcommand*\pandocbounded[1]{% scales image to fit in text height/width
  \sbox\pandoc@box{#1}%
  \Gscale@div\@tempa{\textheight}{\dimexpr\ht\pandoc@box+\dp\pandoc@box\relax}%
  \Gscale@div\@tempb{\linewidth}{\wd\pandoc@box}%
  \ifdim\@tempb\p@<\@tempa\p@\let\@tempa\@tempb\fi% select the smaller of both
  \ifdim\@tempa\p@<\p@\scalebox{\@tempa}{\usebox\pandoc@box}%
  \else\usebox{\pandoc@box}%
  \fi%
}
% Set default figure placement to htbp
\def\fps@figure{htbp}
\makeatother
\setlength{\emergencystretch}{3em} % prevent overfull lines
\providecommand{\tightlist}{%
  \setlength{\itemsep}{0pt}\setlength{\parskip}{0pt}}
\setcounter{secnumdepth}{-\maxdimen} % remove section numbering
\usepackage{booktabs}
\usepackage{graphicx}
\usepackage{float}
\usepackage{fontspec}
\setmainfont{Arial}
\usepackage{placeins}
\usepackage{longtable}
\usepackage{threeparttable}
\usepackage{setspace}
\setstretch{1}
\usepackage{titlesec}
\titleformat{\section}{\Large\bfseries}{\thesection}{1em}{}
\titleformat{\subsection}{\large\bfseries}{\thesubsection}{1em}{}
\usepackage{booktabs}
\usepackage{longtable}
\usepackage{array}
\usepackage{multirow}
\usepackage{wrapfig}
\usepackage{float}
\usepackage{colortbl}
\usepackage{pdflscape}
\usepackage{tabu}
\usepackage{threeparttable}
\usepackage{threeparttablex}
\usepackage[normalem]{ulem}
\usepackage{makecell}
\usepackage{xcolor}
\usepackage{siunitx}

    \newcolumntype{d}{S[
      table-align-text-before=false,
      table-align-text-after=false,
      input-symbols={-,\*+()}
    ]}
  
\usepackage{bookmark}
\IfFileExists{xurl.sty}{\usepackage{xurl}}{} % add URL line breaks if available
\urlstyle{same}
\hypersetup{
  pdftitle={Firm Financial Comparison: 10785 vs.~Berlin},
  pdfauthor={Binghui Ru},
  hidelinks,
  pdfcreator={LaTeX via pandoc}}

\title{Firm Financial Comparison: 10785 vs.~Berlin}
\author{Binghui Ru}
\date{2025-05-28}

\begin{document}
\maketitle

\setstretch{1}
\subsection{1. Research Design and
Methodology}\label{research-design-and-methodology}

This empirical analysis investigates financial differences between firms
located in postal code 10785 and those across Berlin using Orbis panel
data for the years 2019 and 2020. The primary aim is to assess whether
the COVID-19 pandemic has influenced firm-level characteristics by
comparing pre-pandemic (2019) and pandemic (2020) data. The key
financial variables are \textbf{Total Assets} and the \textbf{Equity
Ratio}.

\subsubsection{Three-Stage Analytical
Approach:}\label{three-stage-analytical-approach}

\begin{itemize}
\tightlist
\item
  \textbf{Descriptive Statistics} -- Presenting basic data
  characteristics\\
\item
  \textbf{Difference Testing} -- Including Welch's t-test and Wilcoxon
  non-parametric test\\
\item
  \textbf{Causal Inference} -- Fixed-effects regression models with
  clustered standard errors
\end{itemize}

\subsubsection{Key Variable Definitions}\label{key-variable-definitions}

\begin{itemize}
\tightlist
\item
  \textbf{log toas}: Natural logarithm of total assets
\item
  \textbf{equity ratio winsor}: Equity-to-assets ratio, winsorized at
  1\% and 99\% to control for outliers
\end{itemize}

\newpage

\subsection{2. Descriptive Statistics}\label{descriptive-statistics}

The 2020 data (n = 27,908) indicate substantial variation in firm size.
The mean log total assets is 13.62, and winsorized equity ratio averages
0.142 with a wide spread. Comparing these with 2019 enables assessment
of pandemic-related changes.

\begin{ThreePartTable}
\begin{TableNotes}
\item \textit{Note: } 
\item Data from Berlin firms, 2020. Winsorized equity ratio trims top and bottom 1\% of the distribution.
\end{TableNotes}
\begin{longtable}[t]{lrrrrrrr}
\caption{\label{tab:descriptive-stats}Descriptive Statistics (2020)}\\
\toprule
Variable & n & mean & median & sd & var & min & max\\
\midrule
Log Total Assets & 27908 & 13.624 & 13.705 & 2.321 & 5.386 & 0.000 & 24.904\\
Equity Ratio (Winsorized) & 27908 & 0.142 & 0.367 & 1.389 & 1.929 & -10.434 & 1.000\\
\bottomrule
\insertTableNotes
\end{longtable}
\end{ThreePartTable}

\newpage

\subsection{3. Significance Tests}\label{significance-tests}

We test group differences using both Welch's t-test and Wilcoxon
rank-sum test. The results suggest 10785 firms are significantly smaller
in asset size, but equity ratio differences are less clear. Comparing to
2019 reveals how these gaps evolved due to the pandemic.

\begin{ThreePartTable}
\begin{TableNotes}
\item \textit{Note: } 
\item Welch t-test used for unequal variances. Wilcoxon is a non-parametric alternative.
\end{TableNotes}
\begin{longtable}[t]{lrrrr}
\caption{\label{tab:significance-tests}Significance Tests (2020 Cross-Section)}\\
\toprule
Variable & t-statistic & t p-value & Wilcoxon W & Wilcoxon p-value\\
\midrule
Log Total Assets & 5.576 & 0.00 & 8565835 & 0.000\\
Equity Ratio (Winsorized) & -1.600 & 0.11 & 6898776 & 0.002\\
\bottomrule
\insertTableNotes
\end{longtable}
\end{ThreePartTable}

\subsection{4. Regression Models}\label{regression-models}

Panel regressions confirm that 10785 firms are significantly smaller,
even after controlling for year effects. For equity ratio, models yield
low R\^{}2 and non-significant results, suggesting weak explanatory
power and possible noise.

\FloatBarrier

\begin{table}
\centering
\caption{\label{tab:regression-3a}Regression Results on Log(Total Assets)}
\centering
\begin{tabular}[t]{lcc}
\toprule
  & Log(Total Assets) - Pooled OLS & Log(Total Assets) - Fixed Effects\\
\midrule
(Intercept) & \num{14.236}*** & \num{14.189}***\\
 & (\num{0.000}) & (\num{0.011})\\
groupOther & \num{-0.667}*** & \num{-0.667}***\\
 & (\num{0.059}) & (\num{0.059})\\
factor(year)2020 &  & \num{0.091}***\\
 &  & (\num{0.022})\\
\midrule
Num.Obs. & \num{54024} & \num{54024}\\
R2 & \num{0.002} & \num{0.002}\\
R2 Adj. & \num{0.002} & \num{0.002}\\
RMSE & \num{2.33} & \num{2.33}\\
Std.Errors & by: postcode & by: postcode\\
\bottomrule
\multicolumn{3}{l}{\rule{0pt}{1em}+ p $<$ 0.1, * p $<$ 0.05, ** p $<$ 0.01, *** p $<$ 0.001}\\
\multicolumn{3}{l}{\rule{0pt}{1em}Models use clustered standard errors by postcode.}\\
\multicolumn{3}{l}{\rule{0pt}{1em}Fixed effects include year dummies.}\\
\multicolumn{3}{l}{\rule{0pt}{1em}Dependent variable: log(total assets).}\\
\end{tabular}
\end{table}

\begin{table}
\centering
\caption{\label{tab:regression-3b}Regression Results on Equity Ratio}
\centering
\begin{tabular}[t]{lcc}
\toprule
  & Equity Ratio - Pooled OLS & Equity Ratio - Fixed Effects\\
\midrule
(Intercept) & \num{-1.986}*** & \num{14.451}\\
 & (\num{0.000}) & (\num{17.817})\\
groupOther & \num{-23.533} & \num{-23.557}\\
 & (\num{18.225}) & (\num{18.254})\\
factor(year)2020 &  & \num{-32.099}\\
 &  & (\num{34.794})\\
\midrule
Num.Obs. & \num{54024} & \num{54024}\\
R2 & \num{0.000} & \num{0.000}\\
R2 Adj. & \num{-0.000} & \num{-0.000}\\
RMSE & \num{4125.36} & \num{4125.33}\\
Std.Errors & by: postcode & by: postcode\\
\bottomrule
\multicolumn{3}{l}{\rule{0pt}{1em}+ p $<$ 0.1, * p $<$ 0.05, ** p $<$ 0.01, *** p $<$ 0.001}\\
\multicolumn{3}{l}{\rule{0pt}{1em}Models use clustered standard errors by postcode.}\\
\multicolumn{3}{l}{\rule{0pt}{1em}Fixed effects include year dummies.}\\
\multicolumn{3}{l}{\rule{0pt}{1em}Dependent variable: equity ratio = book equity / total assets.}\\
\end{tabular}
\end{table}

\FloatBarrier

\subsection{5. Conclusion}\label{conclusion}

The empirical results underscore a robust and statistically significant
size disadvantage for firms operating in the 10785 postal region
relative to the general Berlin firm population. This size gap remains
consistent even when comparing the pre-pandemic and pandemic years.
However, no consistent or statistically meaningful difference in equity
ratios was observed over time, which may reflect structural or
measurement heterogeneity not captured in the model. Overall, the
findings offer valuable insights into the spatial and temporal dimension
of corporate financial health within Berlin and provide a starting point
for further research on localized economic resilience during global
crises.

\subsection{6. Reproducibility
Statement}\label{reproducibility-statement}

All empirical results presented in this report were generated using
scripted R code and the original Orbis dataset. The accompanying .Rmd
file and processed tables ensure full reproducibility. Summary
statistics have also been exported to CSV format for transparency and
future research applications.

\end{document}
